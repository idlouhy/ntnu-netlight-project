\section{Introduction}
This document describes deployment and testing process of Wonsole application.
All other information about this software are available in detailed project
report.
\section{Required software}
Some software is required to run the Wonsole. 
\subsection{Web Server}
The client part is a web-application that is server by a web server. Any web
server that can serve static files can fulfill this role. In our installation,
we will use Apache web server.

\subsection{Web Browser}
The user needs a generic web browser to view the files and run the scripts. Any
browser that supports javascript should work. For example refer to Mozilla
Firefox \footnote{\url{http://www.mozilla.org/en-US/firefox}} or Google Chrome
\footnote{\url{https://www.google.com/chrome}}. System has been tested with
Mozilla Firefox $16.0$ and Google Chrome $23.0$. 

\subsection{CouchDB}
The backend server part is represented by CouchDB database system. No other
server software is required, client connects directly to the database system.
The system is tested to work with Apache CouchDB version $1.0.3$ and $1.2$. It
should work with any version supporting the Bulk Documents API
\footnote{\url{http://wiki.apache.org/couchdb/HTTP_Bulk_Document_API}}.





\section{Installation and configuration}
This chapter describes installation and configuration of the tools required to
run this software. As long as you can install all the required software,
choice of the system is up to you. The reference and only supported system by
this manual is be Red Hat Enterprise Linux 6.3.

\subsection{CouchDB}
Easiest way to install Apache CouchDB is to enable the EPEL - Extra Packages for
Enterprise Linux repository. For more information on how to do that please refer
to the offictial guide available on \url{http://fedoraproject.org/wiki/EPEL}.
After setup of the repository, couchdb package should be available for install.
Install this package with all the dependencies.
\begin{verbatim}
yum install couchdb
\end{verbatim}
After installation, we should setup the CouchDB. The default setup file location
is \url{file:///etc/couchdb/local.ini}. We will be using CouchDB directly as a
backend, without a server middleware so we need to set it up to listen on
publicly accesible interface and port. The interface bind address can set using
variable $bind_address$, universal choice is to set it up as $0.0.0.0$ so that
the database listens on all available interfaces. The other interesting option is
the port, that can be set up using the port variable. CouchDB needs to be
available from client side over the network, so choose an interface and port combination, that is allowed on your firewall.

\begin{verbatim}
port = 5984
bind_address = 0.0.0.0
\end{verbatim}

After configuration, start the couchdb service with command
\begin{verbatim}
service couchdb start
\end{verbatim}
You can verify if the service is running using:
\begin{verbatim}
service couchdb status
\end{verbatim}
You should get the feedback similar to
\begin{verbatim}
couchdb (pid  29915) is running...
\end{verbatim}
\subsection{Web Server}
As a reference web server for this installation, we will use Apache $2.2.15$. It
can be installed using command
\begin{verbatim}
yum install httpd
\end{verbatim}
By default, the web server is configured to use the directory /var/www/html/ as
a root for serving the necessary files. Copy all of the files from folder www from
the installation package to folder /var/www/html on your web server.
The web server service httpd can be started using command:
\begin{verbatim}
service httpd start
\end{verbatim}

% http://wiki.apache.org/couchdb/Installation

% \subsection{Same Origin Policy}
% http://stackoverflow.com/questions/3386679/connection-ajax-couchdb-and-javascript http://en.wikipedia.org/wiki/Same_origin_policy
% http://www.w3.org/Security/wiki/Same_Origin_Policy
% \section{Testing}

\section{Appendices}
\subsection{Installation Package Contents}
\begin{verbatim}
wonsole-2.0/
  www/
    css/
    script/
    wonsole.html
  doc
\end{verbatim}

