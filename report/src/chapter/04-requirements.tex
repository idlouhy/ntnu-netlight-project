\chapter{Requirements}
\section{Use cases}
Chosen domain: Library
Possible classes: user/customer, employee, book, author
Use cases: Register new user/customer, Register new employee, Order new book, Add new book to system, List books in the library, Search for specific books, search by author or some other parameter, Reserve a book, Borrow a book / Return a book, Extend a loan, Register state of book when it returns from customer, Remove books from library, Request a book(User), Export inventory, Find placement, Generate reports(Scheduled for return(book) today, etc.)

\section{User stories}
User stories are sentences describing requirements of users and justification of those requirements. They are written in the perspective of the end user.


We compiled list of user stories from a few points of view. Administrator section describes mostly detailed technical requirements. General section contains stories applicable on general problem. Domain section contains domain specific user stories. \footnote{http://scrummethodology.com/scrum-user-stories/}

\subsection*{Developer point of view}
\begin{itemize}
  \item [\textbf{A1}] As a developer, I want to be able to store objects in a persistent database, so that I can migrate data easily.
  \item [\textbf{A2}] As a developer, I want to reflect the changes in persistent storage back to user sessions within one second, so the users always see the latest updated data.
  \item [\textbf{A3}] As a developer, I want to be able to work directly with object attributes rather than any form of raw data.
\end{itemize}

\subsection*{User point of view - General}
\begin{itemize}
  \item [\textbf{G1}] As a user, I want my saved actions to be replicated on to the server within one second, so that my actions will not be lost and the client and server is consistent
  \item [\textbf{G2}] As a user, I want my changes to be propagated to other users of the system real- time
  \item [\textbf{G3}] As a user, I want to be able to see the console and graphical interface at the same time, so that I can use them both side by side simultaneously.
  \item [\textbf{G4}] As a user, I want the changes in console reflect in graphical user interface and likewise, so that I can have overview of the changes I made and I can understand easily, how the system works. In addition this will ensure consistency between the console and graphical interface.
  \item [\textbf{G5}] As a user, I want access to a tutorial, so that I can learn to work with the system easily.
  \item [\textbf{G6}] As a user, I want to be able to display the currently available commands in the console, so I can easily see what I can do with the objects at hand.
  \item [\textbf{G7}] As a user, I want to be able to easily repeat and edit last command, so that I can use it on another object.
  \item [\textbf{G8}] As a user, I want to be able to use batch commands, so that I can work with more than one object at the same time.
\end{itemize}

\subsection*{User point of view - Domain specific}
\begin{itemize}
  \item [\textbf{D1}] As a user, I want to be able to add a new book to the system using both the console and graphical interface.
  \item [\textbf{D2}] As a user, I want to be able to delete a book in the system using both the console and graphical interface.
  \item [\textbf{D3}] As a user, I want to be able to edit information on a specific book and save these changes using both the console and graphical interface.
  \item [\textbf{D4}] As a user, I want to be able to search for a specific book in the system, so that I can watch the information on it using both the console and graphical interface.
  \item [\textbf{D5}] As a user, I want to be able to list all the books currently in the system, so that I easily can get an overview of all the books currently in the library. This should be possible in both the console and the graphical interface.
  \item [\textbf{D6}] As a user, I want to be able to registrate a new customer in the system, so that customers can be saved in the system. This should be possible in both the console and the graphical interface.
  \item [\textbf{D7}] As a user, I want to be able to registrate when a customer borrows a book, so that the information is stored in the system. This should be possible in both the console and the graphical interface.
  \item [\textbf{D8}] As a user, I want to be able to edit the information on a specific borrowing of a book, so that any changes can be recorded. This should be possible in both the console and the graphical interface.
  \item [\textbf{D9}] As a user, I want to be able to list all the books currently borrowed, so that I can get information on each of them. This should be possible in both the console and the graphical interface.
  \item [\textbf{D10}] As a user, I want to be able to reserve a book for a customer, so that customers can request and reserve certain books. This should be possible in both the console and the graphical interface.
  \item [\textbf{D11}] As a user, I want to be able to list all the reservations currently in the system. This should be possible in both the console and the graphical interface.
  \item [\textbf{D12}] As a user, I want to be able to view reservations on specific books, so that I can see if a specific book is available for borrowing at the moment. This should be possible in both the console and the graphical interface.
  \item [\textbf{D13}] As a user, I want to be able to order new books for the library using both the console and the graphical interface.
\end{itemize}



\begin{table}
\begin{tabular}{ | l | l | }
  \hline
  \textbf{user story} & \textbf{difficulty level} \\ \hline
  A1 & 5 \\ \hline
  A2 & 6 \\ \hline
  A3 & 4 \\ \hline
  D1 & 1 (reference) \\ \hline
  D2 & 0.5 \\ \hline
  D3 & 2 \\ \hline
  D4 & 0.75 \\ \hline
  D5 & 1.5 \\ \hline
  D6 & 1 \\ \hline
  D7 & 1.75 \\ \hline
  D8 & 2.5 \\ \hline
  D9 & 2 \\ \hline
  D10 & 3 \\ \hline
  D11 & 2 \\ \hline
  D12 & 1.5 \\ \hline
  D13 & 1.5 \\ \hline
  G1 & 3.5 \\ \hline
  G2 & 2 \\ \hline
  G3 & 8 \\ \hline
  G4 & 4 \\ \hline
  G5 & 3 \\ \hline
  G6 & 3 \\ \hline
  G7 & 2 \\ \hline
  G8 & 3 \\ \hline
\end{tabular}
\caption{User story diffuculty}
\end{table}


\begin{table}
\begin{tabular}{ | l | l | }
  \hline
  \textbf{Sprint 1} & A1,A2,A3,D1,D5 \\ \hline
  \textbf{Sprint 2} &  \\ \hline
  \textbf{Sprint 3} &  \\ \hline
  \textbf{Sprint 4} &  \\ \hline
\end{tabular}
\caption{User story sprint assignment}
\end{table}









\subsection{Planning}
\section{Sequence Diagrams}

\section{Prioritization}



\section{Functional Requirements}
\section{Nonfunctional Requirements}

