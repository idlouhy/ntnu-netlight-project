\chapter{Sprint 4}

\minitoc

This chapter will outline the work we did in the fourth and final sprint. It explains in detail how we planned the sprint, including which user stories we chose and the architecture we employed to implement these. Details on the implementation on each user story is also included, as well documentation on the testing process. Finally our evaluation of the last sprint is presented.

\clearpage


\section{Planning}

\subsection{Duration}
This sprint started on November 5th and lasted for two weeks. A final customer demo was held on the 17th of November to show of the final product for the customer.

\subsection{Sprint Goal}
For this sprint a lot of our focus was put on the report and preparation of the final report. At the beginning at the sprint both the team and the customer agreed that the the product was good and only needed minor improvements for the final presentation. So the goal for this sprint was to get as much work done on the report as possible and at the same time start to prepare much of the final presentation. Also we would like to implement small improvements to the system, making it ready for the final presentation.


\subsection{Product Backlog}
For Sprint 4, the customer primarily wanted us to focus on making the prototype more fit for demonstration purposes, and work on documentation, including the report, a user manual and a deployment manual. 

The user stories completed in sprint 3 were removed, and the following user stories were added:

\begin{itemize}
\item G20 - As a user, I want to be able to add new objects to the system using a simple syntax.
\item G21 - As a user, I want to be able to delete objects from the system using a simple syntax.
\item G22 - As a user, I want to be able to define and execute scripts.
\end{itemize}

The full product backlog for sprint 4 can be found in section \ref{sprint4pb} in the appendix.

\subsection{Sprint Backlog}
The user stories included in the sprint backlog is presented in Table~\ref{table:sp4backlog}. Because the majority of the desired functionality was already implemented, we would spend less time on implementation on this sprint.

\begin{table}
\caption{Sprint 4 Backlog}
\centering
\begin{tabular}{ l p{8cm} l l }
\hline 
			&				&\multicolumn{2}{c}{Hours}			\\
 User Story	& Short Description		&Est.		&Act.	                               \\ 
\hline \\ [-2.0ex]
 
 \bf{G13}	  &\bf{Call function on group of objects}		&\bf{8}		&\bf{9.5}		     \\
		  &Create mechanism for executing code on an object	&			&		\\
		  &Create DSL command						&			&		\\
		  &Testing								&			&		\\
		  &Documentation							&			&		\\

 \bf{G19}     &\bf{Filter a list of objects} 				&\bf{4}		&\bf{4}               \\ 
		  &Create a function to filter a list				&			&		\\
		  &Create DSL command						&			&		\\
		  &Testing								&			&		\\
		  &Documentation							&			&		\\

 \bf{G20}     &\bf{Add new objects with simple syntax} 		&\bf{1}		&\bf{1}		     \\ 
		  &Create DSL command						&			&		\\
		  &Testing								&			&		\\
		  &Documentation							&			&		\\

 \bf{G21}   	&\bf{Delete objects with simple syntax}			&\bf{1}		&\bf{1}		     \\ 
		  &Create DSL command						&			&		\\
		  &Update GUI								&			&		\\
		  &Testing								&			&		\\
		  &Documentation							&			&		\\

 \bf{G21}   	&\bf{Define and execute scripts}				&\bf{8}		&\bf{7}		     \\ 
		  &Create mechanism for storing and executing scripts	&			&		\\
		  &Create DSL command						&			&		\\
		  &Testing								&			&		\\
		  &Documentation							&			&		\\

\hline 
		  &\bf{Total:}						&\bf{22}		&\bf{22.5}		\\
\hline
\end{tabular}
\label{table:sp4backlog}
\end{table}

\section{Architecture}
The user stories picked out for sprint four did not change much on the architecture from the previous sprint. Some rearanging was done, but nothing major.

\section{Implementation}
The base of Wonsole library was implemented in sprint 3, so in this sprint we
could build on that.

\subsection{Foreach}
The implementation of foreach in the last sprint did not support structured
objects - it was not able to iterate over inner objects. We changed the
implementation, so that it uses command with. This way user can write custom
code and just skip dereference through object.

\subsection{Add and remove}
New required functionality was to add and remove documents. We implemented two
commands, that modify the global variable that stores all documents in the
system. When adding, user can define an object directly, or just create an empty
one \verb|{}|. Object is added to docs array and can be commited or rollbacked.
The deletion process works the same.

\subsection{Script}
New functionality this sprint is the script. User can easily define new array
variables in JavaScript. Command script take this variable and evaluates the
array items as Wonsole commands.  User can run the commands in a step mode. That
saves the script contents to command queue and user can just confirm them
pressing enter one by one.



\section{Testing}
\subsection{Test Results}
We performed a total of 4 test cases during this sprint; TID26-29. The results are listed in Table~\ref{table:sp4testresults}. The test cases themselves can be found in the appendix ~\ref{sec:sp4testcases}.

\begin{table}
\caption{Sprint 4 Test Results}
\centering
\begin{tabular}{ l p{13cm} }

\hline 
Item			&Description		\\
\hline \\ [-2.0ex]

\bf{TestID}		&\bf{TID26}			\\
Description	&Call specific functions on group of objects.	\\
Tester		&Øystein Heimark	\\
Date			&15/11 - 2012	\\
Result		&Success				\\
\hline \\ [-2.0ex]

\bf{TestID}		&\bf{TID27}			\\
Description	&Filtering objects.  	\\
Tester		&Øystein Heimark	\\
Date			&15/11 - 2012	\\
Result		&Success			\\
\hline \\ [-2.0ex]

\bf{TestID}		&\bf{TID28}			\\
Description	&Adding and deletion of objects.	\\
Tester		&Øystein Heimark	\\
Date			&15/11 - 2012	\\
Result		&Success\\
\hline \\ [-2.0ex]

\bf{TestID}		&\bf{TID29}			\\
Description	&Creating scripts.	\\
Tester		&Øystein Heimark	\\
Date			&15/11 - 2012	\\
Result		&Success			\\
\hline

\end{tabular}
\label{table:sp4testresults}
\end{table}

\subsection{Test Evaluation}
We did not implement too many user stories in this last sprint, so there were not a whole lot of tests to be performed. All the test cases we did passed without any complications to mention.

\section{Customer Feedback}
The customer was impressed by the new scripting functionality, which was not explicitly requested by him. He was also happy with the results in general, and aware that the last sprint would not involve major changes.

The customer wanted us to document that the prototype allows a lot of unsafe user input in its current state, since it permits any JavaScript code. Work for the future would be to put more restrictions on what the user is allowed to do, to avoid problems.

Our final prototype does not solve a problem that hasn't been solved before; There are existing systems that have similar feature sets. However, it achieves the goals in a more cost efficient way, both for the software developer(us) and the user, and this is where its strength lies.

\section{Evaluation}
\subsection{Review}
We had a very limited sprint backlog for this sprint. Our focus was instead on the project report and demonstration. Regardless, we feel that we did a good job on getting the prototype completed. 
\subsection{Positive}
\begin{itemize}
 \item Implemented all user stories
 \item Discovered new possibilities with the console underway
 \item Implemented more than customer wanted
 \item Useful feedback on trial presentation with advisor
\end{itemize}
\subsection{Negative}
\begin{itemize}
 \item Communication problems with customer (demonstration was unexpectedly postponed one day)
 \item Presentation only partially complete, still a lot of work to do
 \item Poor time management led to a lot of work near the end of the sprint
\end{itemize}