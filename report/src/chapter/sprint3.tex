\chapter{Sprint 3}

\minitoc

\subsubsection{Purpose}

This chapter will outline the work we did in the third sprint. It explains in detail how we planned the sprint, including which user stories we chose and the architecture we employed to implement these. Details on the implementation on each user story is also included, as well documentation on the testing process. Finally our evaluation of the third sprint is presented. 

\clearpage

\section{Planning}

\subsection{Duration}
This sprint started on October 22th and will last for two weeks. A customer demo will be held at 1st of November to show of what we have achieved during the sprint and to ensure that the customer agrees with the solutions and the direction we are going in.

\subsection{Sprint Goal}
The goal for this sprint is to redesign the system to accomodate the new wishes from the customer, and present a working protoype with the suggested technologies.

\subsection{User stories}

\begin{table}
\caption{Sprint 3 User Stories}
\centering
\begin{tabular}{ l p{8cm} l l }
\hline 
			&				&\multicolumn{2}{c}{Hours}			\\
 User Story	& Short Description		&Est.		&Act.	                               \\ 
\hline \\ [-2.0ex]
 
\bf{1}     &\bf{Navigate the application like a directory}		&\bf{5}		&\bf{3.5}          \\ 
		  &Decide on syntax						&			&		\\
		  &Detect and parse navigation commands	&			&		\\
		  &Change directory when command is issued&			&		\\
		  &Testing							&			&		\\
		  &Documentation						&			&		\\

 \bf{2}     &\bf{Store variables in the console} 				&\bf{4}		&\bf{4}               \\ 
		  &Make objects in current directory available		&			&		\\
		  &Ensure persistent storage of the stored objects	&			&		\\
		  &Testing								&			&		\\
		  &Documentation							&			&		\\

 \bf{3}     &\bf{GUI represent current directory} 			&\bf{5}		&\bf{4}		     \\ 
		  &Make viewsfor each directory in CouchDB		&			&		\\
		  &Call the correct view when changing directory	&			&		\\
		  &Update the GUI							&			&		\\
		  &Testing								&			&		\\
		  &Documentation							&			&		\\

 \bf{4}   	&\bf{Expose properties of the objects}			&\bf{6}		&\bf{7.5}		     \\ 
		  &Make objects available						&			&		\\
		  &Make objects in GUI clickable				&			&		\\
		  &Query database on selection				&			&		\\
		  &Update GUI								&			&		\\
		  &Testing								&			&		\\
		  &Documentation							&			&		\\

 \bf{5}	  &\bf{Call specific functions}					&\bf{8}		&\bf{5.5}		     \\
		  &Allow user to extract objects				&			&		\\
		  &Create DSL command						&			&		\\
		  &Testing								&			&		\\
		  &Documentation							&			&		\\

\bf{6}	  &\bf{Perform mathematical operations}			&\bf{8.5}		&\bf{6}		     \\
		  &Allow user to use mathematical functions		&			&		\\
		  &Testing								&			&		\\
		  &Documentation							&			&		\\

\bf{7}   	&\bf{Change or add attributes}				&\bf{3}		&\bf{4}		     \\ 
		  &Allow user to edit the attributes				&			&		\\
		  &Allow user to add new attributes				&			&		\\
		  &Testing								&			&		\\
		  &Documentation							&			&		\\

\bf{8}   	&\bf{Allow user to add entire JSON objects}			&\bf{1.5}		&\bf{2}		     \\ 
		  &Allow user to assign JSON objects as attributes		&			&		\\
		  &Testing									&			&		\\
		  &Documentation								&			&		\\

\bf{9}   	&\bf{Replicate changes to GUI}				&\bf{3}		&\bf{2}		     \\ 
		  &Detect changes made from the console		&			&		\\
		  &Update the GUI accordingly					&			&		\\
		  &Testing								&			&		\\
		  &Documentation							&			&		\\

\bf{10}   	&\bf{Allow user to work locally}				&\bf{2}		&\bf{4}		     \\ 
		  &Keep track of actions made by user			&			&		\\
		  &Add command for storing on server			&			&		\\
		  &Testing								&			&		\\
		  &Documentation							&			&		\\

\hline 
		  &\bf{Total:}						&\bf{46}		&\bf{42.5}		\\
\hline
\end{tabular}
\label{table:sp3usrstories}
\end{table}



\section{Architecture}
\subsection{4+1 view model}
\subsubsection{Logical View}
\subsubsection{Process View}
\subsubsection{Physical View}
\subsubsection{Development View}


\section{Implementation}

\section{Testing}
\subsection{Test Results}
We performed a total of 6 test cases during this sprint; TID20-25. The results are listed in Table~\ref{table:sp3testresults}. The test cases themselves can be found in the appendix ~\ref{sec:sp3testcases}.

\begin{table}
\caption{Sprint 3 Test Results}
\centering
\begin{tabular}{ l p{13cm} }

\hline 
Item			&Description		\\
\hline \\ [-2.0ex]

\bf{TestID}		&\bf{TID20}			\\
Description	&Changing directory in the application in the console.	\\
Tester		&Øystein Heimark	\\
Date			&02/11 - 2012	\\
Result		&Success				\\
\hline \\ [-2.0ex]

\bf{TestID}		&\bf{TID21}			\\
Description	&Storing objects as local variable.  	\\
Tester		&Øystein Heimark	\\
Date			&02/11 - 2012	\\
Result		&Success			\\
\hline \\ [-2.0ex]

\bf{TestID}		&\bf{TID22}			\\
Description	&Allow for the use of functions on the objects.	\\
Tester		&Øystein Heimark	\\
Date			&02/11 - 2012	\\
Result		&Success			\\
\hline \\ [-2.0ex]

\bf{TestID}		&\bf{TID23}			\\
Description	&Allow for editing and adding of attributes.	\\
Tester		&Øystein Heimark	\\
Date			&02/11 - 2012	\\
Result		&Failure. It is not possible to add attributes from the GUI			\\
\hline \\ [-2.0ex]

\bf{TestID}		&\bf{TID24}			\\
Description	&Update GUI according to changes.	\\
Tester		&Øystein Heimark	\\
Date			&02/11 - 2012	\\
Result		&Success		'	\\
\hline \\ [-2.0ex]

\bf{TestID}		&\bf{TID25}			\\
Description	&Store changes to database.\\
Tester		&Øystein Heimark	\\
Date			&02/11 - 2012	\\
Result		&Success			\\
\hline

\end{tabular}
\label{table:sp3testresults}
\end{table}

\subsection{Test Evaluation}
We had one failed test this time, which was due to some missing functionality in the GUI. More specificly it is not possible to add new attributes to existing objects from the GUI. We decided not to correct this, as the focus of this project is the console and its functionality. We prioritized to implement more functionality to the console instead. This was a feeling shared by the customer, they wanted us to exhibit what was possible to do in the console. The functionality of the GUI was not important to them.

\section{Customer Feedback}
This sections covers the feedback we got from the customer, both before and after the sprint.
\newline
\newline
At the end of sprint 2 the customer representative informed us that he would have a meeting with others in the company, to try to identify what our product was missing, a x- factor that would appeal to the users and promote the console. At the Scrum meeting for the third sprint he announced that they had indeed found this x- factor. He wanted us to return to the roots of the project, namely adding scripting to a web- page. We should focus on adding functionality which is impossible or difficult and time consuming to do in a regular GUI.
\newline
\newline
He wanted us to add functionality that would show of the advantages of the technologies we are using, and how it would not be possible to do this with more traditional technologies. He listed some general use cases he wanted us to implement, and suggested that we looked into another solution for the backend. This was because he felt that this solution was better suited for the use cases he now presented us. He was aware of the time constraints of the project and did not expect us to manage to implement all the functionality he mentioned. But it was important that we could document that it would be possible to implement it with the technologies we are using in this project.
\newline
\newline
In the middle of the sprint we did a technology preview for the customer. The purpose of this meeting was to ensure that we were going in the right direction. We presented the user stories we had derived from the new requirements and the customer was all in all happy with these. The customer suggested that we should focus on core functionality, features that will stand out during the presentation of the project. We also presented him what we had implemented so far, including an early implementation of the CouchDB system and an early draft of the GUI. He was impressed what we had managed to implement so far, and encouraged us to keep up the good work.
\newline
\newline
At the end of the sprint we did a sprint demonstration were we presented our results to the customer. He was very happy with our progress, and didn't expect us to be able to get so far so early. For the next sprint he requested that we make a manual for how to use the system(for the users), and a deployment manual for how to set up the system on a new server(for an administrator). He also wanted us to send him a copy of our report, so he could give us some feedback on it.

\section{Evaluation}
\subsection{Review}
\subsection{Positive}
\subsection{Negative}
\subsection{Planned Responses}
