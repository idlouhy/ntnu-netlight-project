\chapter{Introduction}
In this report we will document our work. This includes our work progress, the technologies we used, our research findings and so on. In this intro section we will describe the project, the goals and briefly the involved parties.
    This chapter contains
\begin{itemize}
\item General information about NTNU and Netlight
\item General information about project
\item Contact information on team members
\item Goals
\item Planned effort
\item Schedule of results (milestones, deliverables, sprint deadlines)
\end{itemize}

\section{NTNU}
NTNU (Norwegian University of Science and Technology) has the main responsibility for higher education in Norway. NTNU has a rich and diverse set of educational roads to pursue for instance faculty of architecture, faculty of humanities, faculty of information technology (which is the origin faculty of this course), and many more. There are about 22 000 students at NTNU, and of them about 1800 are exchange students. 
\footnote{\url{http://www.ntnu.no}}

\section{Netlight}
Netlight, our customer, is a Swedish IT- and consulting-firm. Their field of expertise is within IT-management, IT governance, IT-strategy, IT-organisation and IT-research. They deliver independent solutions based on the customers specs. With the broad field of knowledge they can handle whatever tasks presented by their customers. They reach this goal by focusing on competence, creativity and business sense.
\footnote{\url{http://www.netlight.com/en/}}

\section{General information about project}
The project is the making of the course TDT4290 Customer Driven Project. This is a mandatory subject for all 4th year students at IDI and aims to give all its students experience in a customer guided IT-project and the feel of managing a project in a group. The customer assign the group a task which makes the project close to normal working life situation.

This is a proof of concept project. The underlying task is to research and develop a system where power users can benefit from a console.  The concept aims to ease the workload of a power user who is working with object editing, and to see how the efficiency of a console might prove to improve the work. The power user is usually a user who often works with the system over a longer time, and is in depth familiar with the system. We will research already existing systems of this kind, and look at the possibilities and advantages of such a system in a chosen domain.

Library is the chosen domain, and this will be used to explore and test the concept. The library domain is chosen since it possesses a power user, multiple forms which might be effictiviced through a console. This domain also opens the opportunity to test our system on for instance employees on campus, which is important for the proving of the concept.


\section{Contact information}
Contact information on the involved members of this project.
\begin{table}
\centering
\caption{Contact information}
\begin{tabular}{ l  l  l  }
 \textbf{Person} & \textbf{Email} & \textbf{Role} \\ 
\hline \\[-2.0ex]
 Ivo Dlouhy & idlouhy@gmail.com & Team member \\
 Martin Havig & mcmhav@gmail.com & Team member \\
 Øystein Heimark & oystein@heimark.no & Team member \\
 Oddvar Hungnes & mogfen@yahoo.com & Team member \\ 
\hline \\[-2.0ex]
 Peder Kongelf (Netlight) & peder.kongelf@gmail.com & The customer \\
 Stig Lau (Netlight) & stig.lau@gmail.com & The customer \\ 
\hline \\[-2.0ex]
  Meng Zhu (NTNU) & zhumeng@idi.ntnu.no & The advisor \\ 
\hline
\end{tabular}
\end{table}

\subsection{Stakeholders}
The stakeholders in this project is any person or organization which has some interest or is affected by this system\'s development. They together constructs the different restrictions and goals of the project.

\subsubsection[]{The group}
The group role primarily is to meet all requirements presented by the customer and IDI, and are responsible for developing the system.

\subsubsection[]{The customer}
The customer describes and defines the requirements of the project. 

\subsubsection[]{The advisor}
The advisor advises and guides the group through the project.

\subsubsection[]{The librarians}
The librarians is the end user of this concept, and will help to verify the efficiency of it.



\section{Goals}
\begin{enumerate}
  \item Create a working prototype of a system where a scripting console is embedded into a modern web interface. The console should provide access to viewing and modifying the underlying data objects of the system's domain via a DSL.
  \item Investigate the ramifications of the added functionality, in terms of usability and technical aspects.
  \item Provide extensive documentation and a successful presentation of the end product.
\end{enumerate}

\section{Planned Effort}
The course staff recommends us to work 25 person-hours per week and student. This project is estimated for 14 weeks. Since we at the moment have 4 group members in our group, the available effort will be $14*25*4=1400$ person hours including own reading, meetings, lectures, and seminars. The customer requested 5-7 students to handle this project, it is regrettably not likely that we will be supplied by one extra group member, so we must expect some more work hours divided on the four of us.

\section{Schedule of Results}
\subsection {Deliverables}
These are the deliverables and deadlines, that we have to take into account.
\begin {itemize}

\item August 21, Project start

\item October 14, Pre- Delivery: Deliver a copy of the Abstract, Introduction, the Pre-study and the Choice-of Lifecycle-model chapters to the external examiner (censor) and technical writing teacher. Also deliver the outline of the full report (Table of  Contents).

\item November 22, Final Delivery: Project end. Deliver final report and present and demonstrate the final product at NTNU. Four printed and bound copies of  the project report should be delivered, as well as one electronic (PDF) copy.

\end {itemize}

\subsection {Sprints}

Sprint deadlines:
The pre- study, requirements, and testing plan activities should be finished before the start of the first sprint. If this is not the case the number sprints and their deadline might change. The start and end dates of each sprint is listed in Table~\ref{table:sprintdeadlines}

\begin{table}
\caption{Sprint Deadlines}
\centering
\begin{tabular}{ l l l }
\hline
Sprint Nr.		&Start		&End		\\
\hline
1		&24. September		&5. October		\\
2		&8. September			&19. October		\\
3		&22. October			&2. November		\\
4		&4. November			&18. November	\\
\hline
\end{tabular}
\label{table:sprintdeadlines}
\end{table}
