\chapter{Introduction}

\minitoc

Wonsole is a student project under the TDT4290 course in IDI, NTNU. The project is intended to give all its students experience in a customer guided IT- project and the feel of managing a project in a group. Every phase of a typical IT- project will be covered. This report will serve as documentation of our work. This includes our work progress, the technologies we used, our research findings and so on. The introduction chapter is meant to describe the project, our goals and briefly the involved parties.

\clearpage

\section{General information about project}

\subsection{TDT4290 Customer Driven Project}
The project is the making of the course TDT4290 Customer Driven Project. Customer driven project is a course held at NTNU, described in section \ref{NTNU}. Customer driven project is held through one semester, and accounts for 15 credits. This is a mandatory subject for all 4th year students at IDI and aims to give all its students experience in a customer guided IT-project and the feel of managing a project in a group. In this course the students are divided into groups, and each of these groups receives a customer. 
The customer has put together a task for the group to handle, and to make sure that the product produced throughout the project corresponds to the wishes of the customer, the group and the customer should have a close relationship. 
Each group also receives an advisor. This advisor will support the group with inputs and issue solution suggestions. 
The delivery of the course is a report and a product, this will be presented to an examiner. The report is the most important part of the project, and will contain information such as: preliminary study, project management, architecture, conclusion and more. 
The course will provide realistic experience in both report writing and product development driven by a customer. This will help the students perform better when they are out in real life emplyment situations.
\footnote{\url{http://www.idi.ntnu.no/emner/tdt4290/}}

\section{Project}
This is a proof of concept project. The underlying task is to research and develop a system where power users can benefit from a console.  The concept aims to ease the workload of a power user who is working with object editing, and to see how the efficiency of a console might prove to improve the work. The power user is usually a user who often works with the system over a longer time, and is in depth familiar with the system. We will research already existing systems of this kind, and look at the possibilities and advantages of such a system in a chosen domain.

We have chosen a library as our domain, and this will be used to explore and test the concept. The library domain is chosen since it possesses potential for the existence of power users and multiple input forms which could be made more efficient through a console. This domain also opens the opportunity to test our system on for instance employees on campus, which is important for the proving of the concept.
\subsubsection{Goals}
\begin{enumerate}
  \item Provide extensive documentation and a successful presentation of the end product.
  \item Create a working prototype of a system where a scripting console is embedded into a modern web interface. The console should provide access to viewing and modifying the underlying data objects of the system's domain via a DSL.
  \item Investigate the ramifications of the added functionality, in terms of usability and technical aspects.
\end{enumerate}

It is important to note that the report is in focus. It will be the cornerstone of the prototype, to not just ease further development, but also to amplify the reasons for the choices we make.

\section{Stakeholders}
The stakeholders in this project is any person or organization which has some interest or is affected by this system's development. They together constructs the different restrictions and goals of the project.

\subsection{The Team}
The team's role is, primarily, to meet all requirements presented by the customer and IDI. We are responsible for development of the project. Our interest in the project is to receive experience with new technologies and project management, as well as to receive satisfactory grading.
The project was intended for 5-7 students, and the group started out with 5 members, but unfortunately one had to drop the course, which led to a group count of 4. This will, in some way, affect what the team can manage towards what was expected when the project was put together.

\subsection{NTNU}
\label{NTNU}
NTNU (Norwegian University of Science and Technology) has the main responsibility for higher education in Norway. NTNU has a rich and diverse set of educational roads to pursue for instance faculty of architecture, faculty of humanities, faculty of information technology (which is the origin faculty of this course), and many more. There are about 22 000 students at NTNU, and of them about 1800 are exchange students. NTNU’s interest in the project is to provide realistic experiences of high quality in order to educate students, who will later be able to perform better in real life employment situations.
\footnote{\url{http://www.ntnu.no}}

\subsubsection{IDI}
IDI (Department of Computer and Information Science) at NTNU is hosting the Customer Driven Project and will oversee the process and have the final say in determining our grade. IDI is providing an advisor who serves a one-man steering committee and is providing advice and guidance for the group throughout the project.
\footnote{\url{http://idi.ntnu.no}}

\subsection{Netlight}
Netlight, our customer, is a Swedish IT- and consulting-firm. Their field of expertise is within IT-management, IT governance, IT-strategy, IT-organisation and IT-research. They deliver independent solutions based on the customers specs. With the broad field of knowledge they can handle whatever tasks presented by their customers. They reach this goal by focusing on competence, creativity and business sense. 
Peder Kongelf will be our contact person to Netlight. Throughout the project a lot of time will be spent in meetings with him to better understand what is needed to make sure the project is going in the right direction. Netlight's role is to describe and define the requirements of the project.
\footnote{\url{http://www.netlight.com/en/}}

\subsection{Contact information}
Contact information on the involved members of this project.
\begin{table}
\centering
\caption{Contact information}
\begin{tabular}{ l  l  l  }
 \textbf{Person} & \textbf{Email} & \textbf{Role} \\ 
\hline \\[-2.0ex]
 Ivo Dlouhy & idlouhy@gmail.com & Team member \\
 Martin Havig & mcmhav@gmail.com & Team member \\
 Øystein Heimark & oystein@heimark.no & Team member \\
 Oddvar Hungnes & mogfen@yahoo.com & Team member \\ 
\hline \\[-2.0ex]
 Peder Kongelf (Netlight) & peder.kongelf@gmail.com & The customer \\
 Stig Lau (Netlight) & stig.lau@gmail.com & The customer \\ 
\hline \\[-2.0ex]
  Meng Zhu (NTNU) & zhumeng@idi.ntnu.no & The advisor \\ 
\hline
\end{tabular}
\end{table}

\section{Project name}
Project name is important project identificator. It should summarize main project goal or functionality. In real projects, this is often a trademark, or a name that reflects the name of the company. For our project, the main concern was to create a descriptor that reflected the root concept, namely incorporating a console into a web application.

We held a brainstorming meeting specifically to create a name for the project. Early in the process, we created a list of words that could describe our project funcionality or goal. Some keywords:
\begin{itemize}
\item Master, Super User
\item Console, Terminal, Command Line
\item Web Application, GUI
\item Internet, Networking
\item Text, Keyboard
\end{itemize}

From these keywords we attempted to compile a list of candidate names:
\begin{itemize}
\item Console 2.0
\item Wonsole
\item Wensole
\item Websole
\item Werminal
\item interCLI
\end{itemize}

After a discussion and a brief investigation into which names were already taken by other projects, we chose the name \emph{Wonsole}. The project name alone can be a little confusing, so we added the subtitle: \emph{The new web console for power users.}

\section{Structure of Report}
This report describes the development of bringing the scripting experience to the power user. 
The report is structered after the course of the project, and gives the reader insight into the development of the system. 
Chapter 1, the introduction chapter, will introduce the problem to the reader, introduce the members and stakeholders of this project and explain the motivation for doing this poject. After reading this chapter, the reader will be left with an overall view of the projects outline and goals.
Chapter 2, the prestudy chapter, 
Chapter 3, the prestudy chapter, 
Chapter 4, the prestudy chapter, 
Chapter 5, the prestudy chapter, 
Chapter 6, the prestudy chapter, 
Chapter 7, the prestudy chapter, 
Chapter 8, the prestudy chapter, 
Chapter 9, the prestudy chapter, 
Chapter 10, the prestudy chapter, 
Chapter 11, the prestudy chapter, 
Chapter 12, the prestudy chapter, 