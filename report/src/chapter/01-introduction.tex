\chapter{Introduction}

\minitoc

\subsubsection{Purpose}
This report will serve as documentation of our work. This includes our work progress, the technologies we used, our research findings and so on. The introduction chapter is meant to describe the project, our goals and briefly the involved parties.

\clearpage

\section{Stakeholders}

The stakeholders in this project is any person or organization which has some interest or is affected by this system\'s development. They together constructs the different restrictions and goals of the project.

\subsection{The Team}
The team's role is, primarily, to meet all requirements presented by the customer and IDI. We are responsible for development of the project. Our interest in the project is to receive experience with new technologies and project management, as well as to receive satisfactory grading.

\subsection{NTNU}
NTNU (Norwegian University of Science and Technology) has the main responsibility for higher education in Norway. NTNU has a rich and diverse set of educational roads to pursue for instance faculty of architecture, faculty of humanities, faculty of information technology (which is the origin faculty of this course), and many more. There are about 22 000 students at NTNU, and of them about 1800 are exchange students. NTNU's interest in the project is to provide realistic experiences of high quality in order to educate students, who will later be able to perform better in real life employment situations. 
\footnote{\url{http://www.ntnu.no}}

\subsubsection{IDI}
IDI (Department of Computer and Information Science) at NTNU is hosting the Customer Driven Project and will oversee the process and have the final say in determining our grade. IDI is providing an advisor who serves a one-man steering committee and is providing advice and guidance for the group throughout the project.
\footnote{\url{http://idi.ntnu.no}}

\subsection{Netlight}
Netlight, our customer, is a Swedish IT- and consulting-firm. Their field of expertise is within IT-management, IT governance, IT-strategy, IT-organisation and IT-research. They deliver independent solutions based on the customers specs. With the broad field of knowledge they can handle whatever tasks presented by their customers. They reach this goal by focusing on competence, creativity and business sense. Netlight's role is to describe and define the requirements of the project.
\footnote{\url{http://www.netlight.com/en/}}

\subsection{Contact information}
Contact information on the involved members of this project.
\begin{table}
\centering
\caption{Contact information}
\begin{tabular}{ l  l  l  }
 \textbf{Person} & \textbf{Email} & \textbf{Role} \\ 
\hline \\[-2.0ex]
 Ivo Dlouhy & idlouhy@gmail.com & Team member \\
 Martin Havig & mcmhav@gmail.com & Team member \\
 Øystein Heimark & oystein@heimark.no & Team member \\
 Oddvar Hungnes & mogfen@yahoo.com & Team member \\ 
\hline \\[-2.0ex]
 Peder Kongelf (Netlight) & peder.kongelf@gmail.com & The customer \\
 Stig Lau (Netlight) & stig.lau@gmail.com & The customer \\ 
\hline \\[-2.0ex]
  Meng Zhu (NTNU) & zhumeng@idi.ntnu.no & The advisor \\ 
\hline
\end{tabular}
\end{table}


\section{General information about project}
The project is the making of the course TDT4290 Customer Driven Project. This is a mandatory subject for all 4th year students at IDI and aims to give all its students experience in a customer guided IT-project and the feel of managing a project in a group. The customer assign the group a task which makes the project close to normal working life situation.

This is a proof of concept project. The underlying task is to research and develop a system where power users can benefit from a console.  The concept aims to ease the workload of a power user who is working with object editing, and to see how the efficiency of a console might prove to improve the work. The power user is usually a user who often works with the system over a longer time, and is in depth familiar with the system. We will research already existing systems of this kind, and look at the possibilities and advantages of such a system in a chosen domain.

We have chosen a library as our domain, and this will be used to explore and test the concept. The library domain is chosen since it possesses potential for the existence of power users and multiple input forms which could be made more efficient through a console. This domain also opens the opportunity to test our system on for instance employees on campus, which is important for the proving of the concept.

\section{Goals}
\begin{enumerate}
  \item Create a working prototype of a system where a scripting console is embedded into a modern web interface. The console should provide access to viewing and modifying the underlying data objects of the system's domain via a DSL.
  \item Investigate the ramifications of the added functionality, in terms of usability and technical aspects.
  \item Provide extensive documentation and a successful presentation of the end product.
\end{enumerate}

