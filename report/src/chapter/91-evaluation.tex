\chapter{Evaluation}

\minitoc
This is the final phase of this project, the evaluation. Here it will be discussed why and how the outcome ended up being what it is today. This includes how the team worked together, why it gave the result it did, the cooperation with the customer, and how it was working with an overseeing force. Furthermore, we will discuss the issues met during the project, and how the process we used worked for us.

\clearpage

\section{Team Dynamics}
This section will shed light on how the group functioned, the cooperation and how this affected the work.

\subsection*{Work distribution}
We feel like we managed to distribute the work evenly throughout the project and that every group member showed competence in their own fields. The actual work distribution was very dynamic: We agreed on what needed to be done, and then group members had to take individual responsibility to perform work on the tasks they were comfortable with. For our particular group composition we feel that this worked really well and that we were able to deliver great results. However, we realize that with less enthusiastic team members, this model may not have worked as well.

\subsection*{Communication}
In our team, communication was done over skype and through frequent physical meetings. All team members were available online nearly all the time, making coordination a relatively simple task, since other members could be consulted at any time. Furthermore, our small group size allowed us to get well known with each other, leading to high awareness of the other team members’ strengths.

An important point of success in our communication has been conflict resolution. All team members did a good job of bringing up potential points of disagreement so that these could be solved in consensus, or so that temporary solutions could be agreed upon until the matter became more clear.

\subsection*{Effort and estimation}
Our actual amount of logged work fell a little short of the expected amount of hours. However, we were able to successfully execute the tasks that we planned at each step and avoided overworking ourselves. We do not think that the shortage of hours in itself had any significant detrimental effect to the project.

The shortage was largely because of our start phase being slow, due to a sense of confusion about how to get started on the project. We were not given any clear requirements nor existing code or examples to work on. We have learned from this and we will be more confident about taking initiative to push our projects forward in the future.

Another point is that our estimates for the implementation time was steadily a bit higher than the actual implementation time. In reality we think this was a healthy trait that could have helped us compensate for additional implementation problems, should they have appeared.

In general, we think our estimates were within reasonable bounds. This is a difficult task that is not likely to be perfect in any way, but we have definitely gained experience in producing more reasonable estimates.

\section{Customer and project task}
Our customer was very happy with the results. He made it clear that we exceeded his expectations, especially with the fact that we implemented more than he asked for. We experienced some communication problems from time to time, including meeting invitations getting lost in the customer’s mobile device, but we addressed these issues and were able to keep ourselves and the customer updated on the progress.

From our point of view the customer was available, gave us great feedback and guidance throughout the project. One thing to note is that the mastermind behind the project, Stig Lau, was only available on one occasion. Ideally we would have had more direct contact with him, as we believe this would have served to steer us in the right direction even more quickly. However, our main customer contact, Peder Kongelf, was involved and enthusiastic about the project, made large resources within the company available to us, showed us that the project was important to them. We would not have been able to deliver such a great end result if not for the great effort and constant feedback from the customer. 

As stated in the TDT4290 Compendium(2012), 
“The target is a simple proof-of-concept in order to research any changes to the API, the potential of the method as well as evaluate the usability of scripting languages/DSL and API.”

Initially we were considering basing our implementation on an existing system in order to be better able to assess the actual changes required to implement a console in a system. However, during our meeting with Stig Lau, it was made clear that it was not necessary to base our implementation on an existing web-application. 

In the beginning we also wished to involve more users in testing our system’s usability to better evaluate this. The customer originally wanted us to start working on this after the end of Sprint 2, but he then requested a change to a more technology-oriented focus as discussed earlier and in the conclusion. Discovering the potential of the method then became the main priority of the project. We believe we have indeed discovered and documented unique potential, and that this prioritization was in line with the customer’s request. Yet, given more time it would have been interesting to further investigate the other two points(usability and API changes in existing systems).


The project task was an exciting one, giving us the opportunity to work with something new. The technologies we ended up using were cutting edge and we had fun discovering their true potential. The fact that the task was interesting increased our motivation and efforts. The experiences and knowledge we got from this project is something we will carry with us for years to come.

\section{Advisor}
Our advisor has been a steady aid, with meetings on nearly every week throughout the project. He has been available when we needed him and provided valuable feedback on each step of the way, especially on the writing of the project report.

We could have used this resource to a greater extent, especially with regard to design phase documents.

\section{Development Process}
This section will explain the usage of the chosen development methodology, and the good and bad parts with using this methodology.

\subsection*{Using scrum}
The scrum development methodology was chosen early even though the team members were not too familiar with this sort of development process. But during the planning phase it became clear that it was a natural way to go. This was because agile development methodologies have proven to be efficient in prototype projects, where changes are highly likely to happen. It helps reduce the impact of change, which we got to experience first hand for ourselves. 

Instead of the traditional daily physical stand-up scrum meetings we chose to keep all members updated through Skype. There, we shared experience, troubles and progress on a daily basis, but without the overhead of physically having a meeting. We held a demo presentation for both the customer and the advisor at the end of every sprint, this helped us keeping them both updated on our progress, and opened for feedback. Based on this feedback we could make the needed changes to the system, so our next sprint would improve the system in the right direction. In the end we were able to deliver a product that we and the customer were very happy with, which in hindsight can make us say that we were right to choose Scrum. It allowed us to adapt to changing requirements, find misdirections, and ultimately to fulfill the expectations from the customer. Without the adaptation we would have solved a problem, but not the problem the customer wanted us to solve.

\subsection*{Good}
As mentioned above, Scrum allowed us be flexible and made it easy to change the direction the system was taking. This would have been troublesome without an agile development methodology. Traditional methods such as the Waterfall model are rigid and designed to include only one design phase, one implementation phase, etc. This would have stopped us from being able to handle the redirection we did between sprint two and three towards a more dynamic system. Scrum abled us to scale down in each sprint, focusing on smaller tasks. It also made the involvement of all the members in each aspect of the project easy, since Scrum opened a forum where we shared with each other. 

Our execution of the Scrum process was far from perfect, as was to be expected since none of us had any prior experience in using it. But we learned as we went, and feel the experiences we have accumulated in this project will help us in the years to come. Scrum is the development process of choice in most projects concerning IT and we are bound to face this development model in the future. This project provided us with first hand experience in the importance of constantly following up the customer to be able to deliver a product that meets their expectations.

\subsection*{Bad}
The Scrum process imposes a lot of rules, activities and meetings, which was time consuming. The overhead produced by each meeting, each demo presentation, etc, could in some cases have been better spent towards other activities instead. Without all this overhead we would have gotten more time to focus on the implementation, and maybe include more features in the product.


\section{Testing}
We mainly performed three types of testing in this project, unit testing of the code, test cases that were made for each user story and acceptance testing on each user story with the customer. The test cases were performed towards the end of each sprint, and in multiple cases helped us discover minor bugs or shortcomings of the system. We feel like this gave us sufficient test coverage of the different functionality and parts of the system. According to the test plan we were also supposed to do user and system testing, however this was not done. The reason for this is explained below. All in all we are satisfied with our testing process and the amount of testing we were able to do.

We originally planned to include user testing in this project. In the beginning the project was user- centered, focusing on the user experience and usability of the system. It was important for us to involve users in the development process. However as the project developed, it became increasingly technology- centered. The priority shifted to discovering the true potential of the technologies we had chosen. As a result of this change in priorities and the limited time available, user testing was not performed as a part of this project. It is however something we would have liked to do if we were given more time, to test different users reponse to our system and identify improvements that can be made.

Our system turned out to be something else than what we expected from the beginning. The product in its current state is more a platform for developers than a finished product to be set out in production. Because of this it was difficult to identify what to look for in a complete system test of the final product, we did not posses an extensive list of requirements we could base it on. The different components of the system had already been extensively tested at the end of each sprint, both through test cases for each user story and unit tests of the code. Also it would have been difficult to test for specific functionality as the product is able to do pretty much anything. Although it is a library system it does not make sense to test it as a one, as it was never the intended goal for this project to produce a functioning library system. Taking all of this into account we decided not to perform a complete system test on the final product.


\section{Issues}

\subsection*{Group size}
The biggest issue the team encountered was that the team size ended at four. The intended group size for the project was five to seven, which should have produced an extra 325 - 975 work hours, which is a considerable amount of hours. We managed to come out on top of this situation because of a set of responses and effects of having a small group.

\subsubsection*{Ease of communication}
Since we were of such a small size the communication was tight and keeping everyone updated was easy to achieve. This made production efficiency high since there was not done any double work. 

\subsubsection*{The team members}
Taking responsibility came more naturally with such a small group as ours. The members stepped up their game and delivered when it was needed. The general goal of the team was to deliver a good result. This together with a great chemistry between the members made producing a good result with a small group size achievable. 


\subsection*{Changing requirements}
Between the second and the third sprint the requirements from the customer changed somewhat. This opened up for changes to the existing system, and made us reconsider the technologies we had used. This led to a system overhaul, and some bigger changes were made, like changing the database technology we currently used, into another. We managed this issue through some precautions we had made.

\subsubsection*{Risk handling}
Since we were dealing with a prototype project, we always had in mind that requirements could change, and therefore had ease of modifiability in the back of our mind, while developing the system.

\section{Summary}
The course has all in all proved to be a positive and valuable experience for us all. For most of us, the experience of working on such a big project in a team was quite new. We experienced how important it was to plan ahead, to distribute the workload, and to collaborate to achieve a common goal. This was also our first experience in working with an external customer, giving us valuable experiences in this type of project. These experiences are sure to prove useful for in the years to come when we participate in similar projects.

We, as a group, feel that we have reached our goal, and delivered a great product that the customer was very satisfied with. We made our customer happy and exceeded his expectations, and looking back this achievement is something we can be proud of.