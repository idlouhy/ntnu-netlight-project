\chapter{Appendices}
\section{Test Cases}
\subsection{Sprint 1}

\begin{table}
\caption{Test Case TID01}
\centering
\begin{tabular}{ l p{13cm} }
\hline 
 Item            & Description                                                              \\ 
\hline \\ [-2.0ex]
 Description     & Storing objects in a database on the central server                        \\ 
 Tester          & Øystein Heimark                  \\ 
 Preconditions   & There needs to be a server running with a connection to a database available \\ 
 Feature         & Test the ability to store objects permanently on the server from the client  \vspace{3pt}                     \\ 
\hline \\ [-1.5ex]
 Execution steps & \pbox{13cm}{1. Open a new client \\ 2. Call the appropriate method for storing a new object with a given set of attributes from the client. \\ 3. List the content of the database and observe if the new object is indeed stored with its correct attributes. } \vspace{3pt} \\
\hline \\ [-1.5ex]
 Expected result & The object is stored in the database with the correct attributes                                          \\
\hline 
\end{tabular}
\label{table:testcasetid01}
\end{table}


\begin{table}
\caption{Test Case TID02}
\centering
\begin{tabular}{ l p{13cm} }
\hline 
 Item            & Description                                                              \\ 
\hline \\ [-2.0ex]
 Description     & Retrieving objects from the database on the central server \\ 
 Tester          & Øystein Heimark                  \\ 
 Preconditions   & There needs to be a server running with a connection to a database available \\ 
 Feature         & Test the clients ability to retrieve objects from the server   \vspace{3pt}                     \\ 
\hline \\ [-1.5ex]
 Execution steps & \pbox{13cm}{1. Open a new client. \\ 2. Call the appropriate method for retrieving an object. \\ 3. Observe the response from the server. } \vspace{3pt} \\
\hline \\ [-1.5ex]
 Expected result & The object is successfully retrieved from the server with the correct attributes          \\
\hline 
\end{tabular}
\label{table:testcasetid02}
\end{table}


\begin{table}
\caption{Test Case TID03}
\centering
\begin{tabular}{ l p{13cm} }
\hline 
 Item            & Description                                                              \\ 
\hline \\ [-2.0ex]
 Description     & Sending real- time messages from server to client \\ 
 Tester          & Øystein Heimark                  \\ 
 Preconditions   & There needs to be a server able to send messages up and running, and a client ready to receive \\ 
 Feature         & Test the ability to send real- time messages from server to client   \vspace{3pt}                     \\ 
\hline \\ [-1.5ex]
 Execution steps & \pbox{13cm}{1. Open a new client. \\ 2. Send a message from the server with the associated method. \\ 3. Observe the output on the client side. } \vspace{3pt} \\
\hline \\ [-1.5ex]
 Expected result & The message will be received by the client and displayed within one second from when the message is sent from the server.          \\
\hline 
\end{tabular}
\label{table:testcasetid03}
\end{table}

\begin{table}
\caption{Test Case TID04}
\centering
\begin{tabular}{ l p{13cm} }
\hline 
 Item            & Description                                                              \\ 
\hline \\ [-2.0ex]
 Description     & Alerting clients that there has been added a book to the central database on the server \\ 
 Tester          & Øystein Heimark                  \\ 
 Preconditions   & TID03 and either TID05 or TID06 must alredady have passed. The server must be running \\ 
 Feature         &The ability to alert multiple clients that a new book is added to the system real- time \vspace{3pt}                     \\ 
\hline \\ [-1.5ex]
 Execution steps & \pbox{13cm}{1. Open the application with multiple clients. \\ 2. Add a new book from one of the clients. \\ 3. Observe the output on all the clients} \vspace{3pt} \\
\hline \\ [-1.5ex]
 Expected result & All the clients will be alerted within one second that a new book has been added, and the list of books in the client will be updated.          \\
\hline 
\end{tabular}
\label{table:testcasetid04}
\end{table}


\begin{table}
\caption{Test Case TID05}
\centering
\begin{tabular}{ l p{13cm} }
\hline 
 Item            & Description                                                              \\ 
\hline \\ [-2.0ex]
 Description     & Verifying that domain specific objects are available through the console \\ 
 Tester          & Øystein Heimark                  \\ 
 Preconditions   &A console must be available \\ 
 Feature         &The ability to work directly with domain specific objects and objects attributes \vspace{3pt}                     \\ 
\hline \\ [-1.5ex]
 Execution steps & \pbox{13cm}{1. Open a console. \\ 2. Create a book object. \\ 3. Change the attribute of the newly created object by command.} \vspace{3pt} \\
\hline \\ [-1.5ex]
 Expected result & The user is able to retrieve objects and change their attributes via the console.          \\
\hline 
\end{tabular}
\label{table:testcasetid05}
\end{table}


\begin{table}
\caption{Test Case TID06}
\centering
\begin{tabular}{ l p{13cm} }
\hline 
 Item            & Description                                                              \\ 
\hline \\ [-2.0ex]
 Description     & Verifying that there is a console and graphical interface present on each page\\ 
 Tester          & Øystein Heimark                  \\ 
 Preconditions   &None \\ 
 Feature         &Simultaneous display of console and graphical interface \vspace{3pt}                     \\ 
\hline \\ [-1.5ex]
 Execution steps & \pbox{13cm}{1. Open a new instance of the application with a web- client. \\ 2. Observe if there is a graphical interface as well as a console present.} \vspace{3pt} \\
\hline \\ [-1.5ex]
 Expected result & Console and graphical interface is present on the same page.          \\
\hline 
\end{tabular}
\label{table:testcasetid06}
\end{table}


\begin{table}
\caption{Test Case TID07}
\centering
\begin{tabular}{ l p{13cm} }
\hline 
 Item            & Description                                                              \\ 
\hline \\ [-2.0ex]
 Description     & Adding a new book to the system with the graphical web- application \\ 
 Tester          & Øystein Heimark                  \\ 
 Preconditions   & The server with the REST api must be running. A graphical interface must be available. \\ 
 Feature         & The ability to add new books to the system from a client with the graphical web- application   \vspace{3pt}                     \\ 
\hline \\ [-1.5ex]
 Execution steps & \pbox{13cm}{1. Open the application with a web client \\ 2. Add a new book from the web- application on the client. \\ 3. List the books currently on the system and observe if the new book is added.} \vspace{3pt} \\
\hline \\ [-1.5ex]
 Expected result & The new book is added to the system and the list of books with the attributes stated in the creation of the book.          \\
\hline 
\end{tabular}
\label{table:testcasetid07}
\end{table}



\begin{table}
\caption{Test Case TID08}
\centering
\begin{tabular}{ l p{13cm} }
\hline 
 Item            & Description                                                              \\ 
\hline \\ [-2.0ex]
 Description     & Adding a new book to the system with the console. A console must be available \\ 
 Tester          & Øystein Heimark                  \\ 
 Preconditions   & The server with the REST api must be running \\ 
 Feature         & The ability to add new books to the system from a client with the console.   \vspace{3pt}                     \\ 
\hline \\ [-1.5ex]
 Execution steps & \pbox{13cm}{1. Open the application with a web client \\ 2. Add a new book from the console on the client. \\ 3. Observe the list of the books currently on the system and observe if the new book is in this list.} \vspace{3pt} \\
\hline \\ [-1.5ex]
 Expected result & The new book is added to the system and the list of books with the attributes stated in the creation of the book.          \\
\hline 
\end{tabular}
\label{table:testcasetid08}
\end{table}


\begin{table}
\caption{Test Case TID09}
\centering
\begin{tabular}{ l p{13cm} }
\hline 
 Item            & Description                                                              \\ 
\hline \\ [-2.0ex]
 Description     & Listing all the books currently in the system using the graphical web- application \\ 
 Tester          & Øystein Heimark                  \\ 
 Preconditions   & The server with the REST api must be running. There has to be books stored in the database. A graphical interface must be available\\ 
 Feature         & The ability to get an overview of the books currently in the system using the web- application   \vspace{3pt}                     \\ 
\hline \\ [-1.5ex]
 Execution steps & \pbox{13cm}{1. Obtain a list of all the books in the system directly from the central database/server \\ 2. Use the graphical web- application to get a list of all the books in the system \\ 3. Compare the result from step two to the one obtained in step 1, and verify that they contain the same books.} \vspace{3pt} \\
\hline \\ [-1.5ex]
 Expected result & The list of books presented in the graphical web- application is identical to the one stored on the central database/server.          \\
\hline 
\end{tabular}
\label{table:testcasetid09}
\end{table}


\begin{table}
\caption{Test Case TID10}
\centering
\begin{tabular}{ l p{13cm} }
\hline 
 Item            & Description                                                              \\ 
\hline \\ [-2.0ex]
 Description     & Listing all the books currently in the system using the console. \\ 
 Tester          & Øystein Heimark                  \\ 
 Preconditions   & The server with the REST api must be running. There has to be books stored in the database.  A console must be available\\ 
 Feature         & The ability to get an overview of the books currently in the system using console.   \vspace{3pt}                     \\ 
\hline \\ [-1.5ex]
 Execution steps & \pbox{13cm}{1. Obtain a list of all the books in the system directly from the central database/server \\ 2. Use the console to get a list of all the books in the system \\ 3. Compare the result from step two to the one obtained in step 1, and verify that they contain the same books.} \vspace{3pt} \\
\hline \\ [-1.5ex]
 Expected result & The list of books presented in the console is identical to the one stored on the central database/server.          \\
\hline 
\end{tabular}
\label{table:testcasetid10}
\end{table}


\subsection{Sprint 2}


\begin{table}
\caption{Test Case TID11}
\centering
\begin{tabular}{ l p{13cm} }
\hline 
 Item            & Description                                                              \\ 
\hline \\ [-2.0ex]
 Description     &Storing objects without a schema in a database on the central server . \\ 
 Tester          & Øystein Heimark                  \\ 
 Preconditions   & There needs to be a server up and running with a database available\\ 
 Feature         & The ability to store objects with a different attribute set, in the same database.   \vspace{3pt}                     \\ 
\hline \\ [-1.5ex]
 Execution steps & \pbox{13cm}{1. Call the appropriate method for storing a new object with a given set of attributes \\ 2. Call the same method again, but provide an object with a different set of attributes \\ 3. Observe that both objects are stored in the database with the correct attributes.} \vspace{3pt} \\
\hline \\ [-1.5ex]
 Expected result & Both objects, with different attributes, are stored in the database.          \\
\hline 
\end{tabular}
\label{table:testcasetid11}
\end{table}


\begin{table}
\caption{Test Case TID12}
\centering
\begin{tabular}{ l p{13cm} }
\hline 
 Item            & Description                                                              \\ 
\hline \\ [-2.0ex]
 Description     &Printing out commands in the console while operating with the GUI. \\ 
 Tester          & Øystein Heimark                  \\ 
 Preconditions   & None\\ 
 Feature         & For every action made in the GUI the corresponding command in the console should be printed in the console.   \vspace{3pt}                     \\ 
\hline \\ [-1.5ex]
 Execution steps & \pbox{13cm}{1. Open a new client \\ 2. Do a lot of different actions in the GUI. \\ 3. Observe that the correct commands are printed in the console.} \vspace{3pt} \\
\hline \\ [-1.5ex]
 Expected result & The correct commands are printed in the console.          \\
\hline 
\end{tabular}
\label{table:testcasetid12}
\end{table}


\begin{table}
\caption{Test Case TID13}
\centering
\begin{tabular}{ l p{13cm} }
\hline 
 Item            & Description                                                              \\ 
\hline \\ [-2.0ex]
 Description     &Showing a popup menu in the console with the available methods and attributes for the object which is currently selected. \\ 
 Tester          & Øystein Heimark                  \\ 
 Preconditions   & None\\ 
 Feature         & The ability to list the methods and attributes of a given object in a popup menu.   \vspace{3pt}                     \\ 
\hline \\ [-1.5ex]
 Execution steps & \pbox{13cm}{1. Open a new client \\ 2. Create a new object. \\ 3. Select that object using the console. \\ 4. Show the popup menu using the corresponding hotkey} \vspace{3pt} \\
\hline \\ [-1.5ex]
 Expected result & The correct methods and attributes of the selected object is shown in the popup menu.          \\
\hline 
\end{tabular}
\label{table:testcasetid13}
\end{table}


\begin{table}
\caption{Test Case TID14}
\centering
\begin{tabular}{ l p{13cm} }
\hline 
 Item            & Description                                                              \\ 
\hline \\ [-2.0ex]
 Description     &Selecting an element from the popup menu and insert the selected method or attribute in the console. \\ 
 Tester          & Øystein Heimark                  \\ 
 Preconditions   & None\\ 
 Feature         & The ability to autocomplete methods and attributes selected from the popup menu.   \vspace{3pt}                     \\ 
\hline \\ [-1.5ex]
 Execution steps & \pbox{13cm}{1. Open a new client \\ 2. Select an object using the console. \\ 3. Pick a method or attribute from the popup menu. } \vspace{3pt} \\
\hline \\ [-1.5ex]
 Expected result & The selected method or attribute from the popup menu is printed in the console.          \\
\hline 
\end{tabular}
\label{table:testcasetid14}
\end{table}


\begin{table}
\caption{Test Case TID15}
\centering
\begin{tabular}{ l p{13cm} }
\hline 
 Item            & Description                                                              \\ 
\hline \\ [-2.0ex]
 Description     &Highlighting an object in the GUI when it is selected from the console. \\ 
 Tester          & Øystein Heimark                  \\ 
 Preconditions   & None\\ 
 Feature         & Highlighting a selected object.   \vspace{3pt}                     \\ 
\hline \\ [-1.5ex]
 Execution steps & \pbox{13cm}{1. Open a new client \\ 2. Select an object using the console. \\ 3. Observe the response in the GUI. } \vspace{3pt} \\
\hline \\ [-1.5ex]
 Expected result & The selected object should be highlighted in the GUI.          \\
\hline 
\end{tabular}
\label{table:testcasetid15}
\end{table}


\begin{table}
\caption{Test Case TID16}
\centering
\begin{tabular}{ l p{13cm} }
\hline 
 Item            & Description                                                              \\ 
\hline \\ [-2.0ex]
 Description     &Highlighting a group of objects in the GUI when it is selected from the console. \\ 
 Tester          & Øystein Heimark                  \\ 
 Preconditions   & The system must be capable of selecting multiple objects\\ 
 Feature         & Highlighting groups of objects.   \vspace{3pt}                     \\ 
\hline \\ [-1.5ex]
 Execution steps & \pbox{13cm}{1. Open a new client \\ 2. Select a group of objects. \\ 3. Observe the response in the GUI. } \vspace{3pt} \\
\hline \\ [-1.5ex]
 Expected result & The selected objects should be highlighted in the GUI.          \\
\hline 
\end{tabular}
\label{table:testcasetid16}
\end{table}


\begin{table}
\caption{Test Case TID17}
\centering
\begin{tabular}{ l p{13cm} }
\hline 
 Item            & Description                                                              \\ 
\hline \\ [-2.0ex]
 Description     &Cycling through the current selection of objects using a hotkey. \\ 
 Tester          & Øystein Heimark                  \\ 
 Preconditions   & The system must be capable of selecting multiple objects\\ 
 Feature         & The ability to cycle through the current selection of objects in the console using a hotkey.   \vspace{3pt}                     \\ 
\hline \\ [-1.5ex]
 Execution steps & \pbox{13cm}{1. Open a new client \\ 2. Select a group of objects. \\ 3. Cycle through the objects using the hotkey in the console. } \vspace{3pt} \\
\hline \\ [-1.5ex]
 Expected result & The objects in the current selection are made available to the user one by one, in the correct order.          \\
\hline 
\end{tabular}
\label{table:testcasetid17}
\end{table}


\begin{table}
\caption{Test Case TID18}
\centering
\begin{tabular}{ l p{13cm} }
\hline 
 Item            & Description                                                              \\ 
\hline \\ [-2.0ex]
 Description     &Highlighting the selected object while cycling through a selection of objects. \\ 
 Tester          & Øystein Heimark                  \\ 
 Preconditions   & The system must be capable of selecting multiple objects\\ 
 Feature         & While cycling through a selection of objects in the console the current object will be highlighted in the GUI.   \vspace{3pt}                     \\ 
\hline \\ [-1.5ex]
 Execution steps & \pbox{13cm}{1. Open a new client \\ 2. Select a group of objects. \\ 3. Cycle through the objects using the hotkey in the console. \\ 4. Observe the response in the GUI. } \vspace{3pt} \\
\hline \\ [-1.5ex]
 Expected result & The highlighted object in the GUI will be updated as you cycle through the selection.          \\
\hline 
\end{tabular}
\label{table:testcasetid18}
\end{table}

\clearpage

\begin{table}
\caption{Test Case TID19}
\centering
\begin{tabular}{ l p{13cm} }
\hline 
 Item            & Description        \\ 
\hline \\ [-2.0ex]
 Description     &Update a separate section of the GUI when the user clicks on a record, and make the user able to edit the record in this section. \\ 
 Tester          & Øystein Heimark                  \\ 
 Preconditions   & None\\ 
 Feature         & The ability to edit the information on a record from a separate section of the GUI.   \vspace{3pt}                     \\ 
\hline \\ [-1.5ex]
 Execution steps & \pbox{13cm}{1. Open a new client \\ 2. Click on a record. \\ 3. Edit the info on the clicked record in the separate section. \\ 4. Observe the response in the GUI. } \vspace{3pt} \\
\hline \\ [-1.5ex]
 Expected result & A separate section of the GUI is updated with the information on the clicked record. When this information is edited the record is updated. \\
\hline 
\end{tabular}
\label{table:testcasetid19}
\end{table}

