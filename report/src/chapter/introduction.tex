\chapter{Introduction}
In this report we will document our work. This includes our work progress, the technologies we used, our research findings and so on. In this intro section we will describe the project, the goals and briefly the involved parties.
    This chapter contains
\begin{itemize}{labelitemi}{$\bullet$}
\item General information about ntnu and netlight
\item General information about project
\item Contact information on team members
\item Goals
\item Planned effort
\item Schedule of results(Milestones, deliverables, sprint deadlines, etc)
\end{itemize}

\section{NTNU}
NTNU (Norwegian University of Science and Technology) has the main responsibility for higher education in Norway. NTNU has a rich and diverse set of educational roads to pursue for instance faculty of architecture, faculty of humanities, faculty of information technology (which is the origin faculty of this course), and many more. There are about 22 000 students at NTNU, and of them about 1800 are exchange students. 
\footnote{\url{http://www.ntnu.no}}

\section{Netlight}
Netlight, our customer, is a Swedish IT- and consulting-firm. Their field of expertise is within IT-management, IT governance, IT-strategy, IT-organisation and IT-research. They deliver independent solutions based on the customers specs. With the broad field of knowledge they can handle whatever tasks presented by their customers. They reach this goal by focusing on competence, creativity and business sense.
\footnote{\url{http://www.netlight.com/en/}}
%\footnote{http://en.wikipedia.org/wiki/Netlight_Consulting}

\section{General information about project}
The project is the making of the course TDT4290 Customer Driven Project. This is a mandatory subject for all 4th year students at IDI and aims to give all its students experience in a customer guided IT-project and the feel of managing a project in a group. The customer assign the group a task which makes the project close to normal working life situation.

This is a proof of concept project. The underlying task is to research and develop a system where power users can benefit from a console.  The concept aims to ease the workload of a power user who is working with object editing, and to see how the efficiency of a console might prove to improve the work. The power user is usually a user who often works with the system over a longer time, and is in depth familiar with the system. We will research already existing systems of this kind, and look at the possibilities and advantages of such a system in a chosen domain.


\section{Contact information}
\begin{tabular}{ | l | l | l | }
  \hline
  \textbf{Person} & \textbf{Email} & \textbf{Role} \\ \hline
  Ivo Dlouhy & idlouhy@gmail.com & Team member \\ \hline
  Martin Havig & mcmhav@gmail.com & Team member \\ \hline
  Oystein Heimark & oystein@heimark.no & Team member \\ \hline
  Oddvar Hungnes & mogfen@yahoo.com & Team member \\ \hline
  Peder Kongelf & peder.kongelf@gmail.com & The customer \\ \hline
  Stig Lau & stig.lau@gmail.com & The customer \\ \hline
  Meng Zhu & zhumeng@idi.ntnu.no & The advisor \\ \hline
\end{tabular}


\section{Goals}
\begin{enumerate}
  \item Create a working prototype of a system where a scripting console is embedded into a modern web interface. The console should provide access to viewing and modifying the underlying data objects of the system's domain via a DSL.
  \item Investigate the ramifications of the added functionality, in terms of usability and technical aspects.
  \item Provide extensive documentation and a successful presentation of the end product.
\end{enumerate}

\section{Planned Effort}
