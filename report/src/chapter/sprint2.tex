\chapter{Sprint 2}
\section{Planning}

\subsection{Duration}
This sprint started on October 8th and will last for two weeks. A customer demo will be held at October 18th to show of what we have achieved during the sprint and to ensure that the customer agrees with the solutions  and the direction we are going in.

\subsection{Sprint Goal}
The goal of the second sprint is to increase the functionality of the console. We will try to include user stories that add functionality that is easy to use and is valuable for the end user. The user stories concerned with domain specifics of the system, like implemeting support for adding users to the library, will be left untouched in this sprint.

\subsection{User Stories}
As of this sprint we tried to split each user story into smaller sub- tasks that could be distributed easily amonst the group members. Each estimate for the subtasks includes design and implementation of the solution. Testing includes writing unit tests as we code, write test cases and to perform them. Documentation includes all the work necessary to document our work on that user story in the report.

For this sprint we had 110 hours available for the sprint, exluding work on the report, project management, sprint planning and evaluation. Its a bit less than normal as we decided to put in some extra work on the pre- delivery of the report on October 14th.

\begin{table}
\caption{Sprint 2 User Stories}
\centering
\begin{tabular}{ l p{8cm} l l }
\hline 
			&				&\multicolumn{2}{c}{Hours}			\\
 User Story	& Short Description		&Est.		&Act.	                               \\ 
\hline \\ [-2.0ex]
 \bf{A4}     &\bf{Update to schemaless database}		&\bf{10}		&\bf{}          \\ 
		  &Change the server implemenation		&4			&		\\
		  &Update REST API calls					&2			&		\\
		  &Testing							&2			&		\\
		  &Documentation						&2			&		\\

 \bf{G4}     &\bf{Reflect actions from the GUI in the console} 		&\bf{11}		&\bf{}               \\ 
		  &Print the commands in the console				&6			&5		\\
		  &Testing									&3			&3		\\
		  &Documentation								&2			&		\\

 \bf{G7}     &\bf{Autocompletion of commands} 	&\bf{30}		&\bf{}		     \\ 
		  &Add methods to list commands		&4			&		\\
		  &Create popup menu				&10			&		\\
		  &Update elements in menu			&4			&		\\
		  &Make selection of commands possible&3			&		\\
		  &Autocomplete on selection			&1			&		\\
		  &Testing						&4			&		\\
		  &Documentation					&4			&		\\

 \bf{G10}   &\bf{Indicate the selected objects in GUI}		&\bf{26}		&\bf{}		     \\ 
		  &Design soulution for highlighting elements	&2			&		\\
		  &Create method for highlighting				&2			&		\\
		  &Create record for selected objects			&4			&		\\
		  &Respond to changes in selection				&8			&		\\
		  &Highlight newly created objects				&2			&		\\
		  &Testing								&4			&		\\
		  &Documentation							&4			&		\\

 \bf{G11}	  &\bf{Cycle through the current selection}	&\bf{25}		&\bf{}		     \\
		  &Logic to keep track of selection			&5			&		\\
		  &Update this list when selection changes	&4			&		\\
		  &Extract objects when cycling through		&7			&		\\
		  &Update the GUI when selection changes	&2			&		\\
		  &Testing							&4			&		\\
		  &Documentation						&3			&		\\

\bf{G8}	  &\bf{Show details on click}			&\bf{8}		&\bf{}		     \\
		  &Make ID clickable					&1			&		\\
		  &Update relevant section on click		&2			&		\\
		  &Able the user to make changes		&3			&		\\
		  &Testing						&1			&		\\
		  &Documentation					&1			&		\\
\hline 
		  &\bf{Total:}						&\bf{110}		&\bf{}		\\
\hline
\end{tabular}
\label{table:sp2usrstories}
\end{table}

\begin{table}
\caption{Sprint 2 Workload}
\centering
\begin{tabular}{ l l l }
\hline 
            &\multicolumn{2}{c}{Hours}          \\
 Task       &Est.           &Act.                                  \\ 
\hline \\ [-2.0ex]
Design          &43     &     \\
Implementation  &63     &     \\
Testing         &19     &     \\
Documentation   &15     &     \\
\hline
\bf{Total}          &\bf{140}   &       \\
\hline
\end{tabular}
\label{table:sp2workload}
\end{table}

\section{Architecture}
\subsection{4+1 view model}
\subsubsection{Logical View}
\subsubsection{Development View}
\subsubsection{Process View}
\subsubsection{Physical View}

\section{Implementation}


\section{Testing}
This section will present the tests performed during the second sprint and their result.

\subsection{Test Results}
We performed a total of 9 test cases during this sprint; TID11-19. The results are listed in Table~\ref{sp2testresults}. The test cases themselves can be found in appendix(TBA).

\begin{table}
\caption{Sprint 2 Test Results}
\centering
\begin{tabular}{ l p{13cm} }

\hline 
Item			&Description		\\
\hline \\ [-2.0ex]

\bf{TestID}		&\bf{TID11}			\\
Description	&Storing objects on the server without a schema	\\
Tester		&Øystein Heimark	\\
Date			&19/10 - 2012	\\
Result		&				\\
\hline \\ [-2.0ex]

\bf{TestID}		&\bf{TID12}			\\
Description	&Printing commands in the console 	\\
Tester		&Øystein Heimark	\\
Date			&19/10 - 2012	\\
Result		&Success			\\
\hline \\ [-2.0ex]

\bf{TestID}		&\bf{TID13}			\\
Description	&Showing the popup menu	\\
Tester		&Øystein Heimark	\\
Date			&19/10 - 2012	\\
Result		&Failure. The popup menu was displayed, but some of the attributes and methods were missing from the menu				\\
\hline \\ [-2.0ex]

\bf{TestID}		&\bf{TID14}			\\
Description	&Selecting elements from the popup menu	\\
Tester		&Øystein Heimark	\\
Date			&19/10 - 2012	\\
Result		&Success			\\
\hline \\ [-2.0ex]

\bf{TestID}		&\bf{TID15}			\\
Description	&Highlighting selected objects	\\
Tester		&Øystein Heimark	\\
Date			&19/10 - 2012	\\
Result		&Failure. Objects are not automaticly highlighted when they selected from the console. It is however possible to highlight an object from the console using a method call.	\\
\hline \\ [-2.0ex]

\bf{TestID}		&\bf{TID16}			\\
Description	&Highlighting a group of objects\\
Tester		&Øystein Heimark	\\
Date			&19/10 - 2012	\\
Result		&Success			\\
\hline \\ [-2.0ex]

\bf{TestID}		&\bf{TID17}			\\
Description	&Cycling through current selection	\\
Tester		&Øystein Heimark	\\
Date			&19/10 - 2012	\\
Result		&Failure. It is not possible to highlight multiple objects.		\\
\hline \\ [-2.0ex]

\bf{TestID}		&\bf{TID18}			\\
Description	&Highlighting while cycling through objects	\\
Tester		&Øystein Heimark	\\
Date			&19/10 - 2012	\\
Result		&Success	\\
\hline \\ [-2.0ex]

\bf{TestID}		&\bf{TID19}			\\
Description	&Update separate section on clicking a book	\\
Tester		&Øystein Heimark	\\
Date			&19/10 - 2012	\\
Result		&Failure. It is not possible to edit the title of the book in the separate section		\\
\hline

\end{tabular}
\label{table:sp2testresults}
\end{table}

\subsection{Test Evaluation}
Some of our test failed this time. Most of them was due to minor bugs in the system which was easily corrected at once. The remaining failures was due to some missing functionality which was defined by the user stories and where yet to be implemented at the time the testing was performed. However we were able to add this functionality and pass all the tests before the beginning of the third sprint.

\section{Customer Feedback}

\subsection{Before}
As mentioned in sprint 1, the customer had a lot of suggestions on what we should implement in this sprint, and gave us some user stories that he wanted us to implement. He suggested that our focus for this sprint should be on improving the functionality of the console, something the group also wanted to do. 

\subsection{After}
The customer was happy with our work on the second sprint, but at the same time felt that the system was still missing something. The workflow was not as fluid as it should be, the console needed a X- factor that appealed to the user. The focus should be on adding functionality that adds value to the console. At the moment the console and its features are very programmatic. The customer told us that he would meet with other representatives from the company to talk about the direction they wanted the project to take from here on, and to maybe find the X- factor we were looking for. We would get this feedback for the start of the third sprint.

The customer also requested that we started to test our product on our peers, as feedback from users would be extremly valuable at this point of the project. He suggested that we should devote soem time in the next sprint to make and perform the user testing, something the group also felt would be a good idea.

\section{Evaluation}

\subsection{Review}

\subsection{Positive}

\begin{itemize}
\item Our effort to improve the time tracking succeded, and everyone loged their time in a good way. This made it easier for the team leader to oversee the progress of the sprint, and it also made it easier to document the work in the report.
\item In this sprint we made a more detailed work breakdown structure, and we really feel this was worth the effort. It allowed us to distribute workload in a better way, and ultimately finish the work that we planned to do.
\item The customer was very much included in the planning of the sprint this time, more than in the last sprint. This allowed them to have a greater influnce on what was to be included in this sprint.
\item The testing proved to be more useful this time. We uncovered errors in the system during the testing process, and parts of the user stories not yet implemented.
\end{itemize}

\subsection{Negative}

\begin{itemize}
\item We can still improve in utilizing all the aspects of the Scrum process. We managed to organize the daily meetings this time, but we think we can improve this even further.
\item The pre- delivery took up a lot of time in this sprint, more than we planned for. We underestimated the effort we had to put in to this.
\item We had some technical issues during the demo which made it difficult for us to perform a proper demosntration of the features we had implemented. The technical aspect of the demo should be prepared better, and we should prepare a backup solution if we happen to experience the same problems in the following demos.
\item We should have performed the testing earlier in the sprint, so we could have uncovered the errors earlier. This way we will be able to correct them before the demo with the customer.
\end{itemize}

\subsection{Planned Responses}
\begin{itemize}
\item We will put more focus on the testing process, and perform the tests earlier. We will then manage to fix the issues in time.
\item Put more effort in doing the daily Scrum meetings.
\item Properly prepare more for the demo, make it easier for the customer to follow the workflow.
\end{itemize}
